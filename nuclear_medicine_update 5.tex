\documentclass[12pt]{article}
% \usepackage[utf8x]{inputenc}

% \usepackage{natbib}

\title{Nuclear Medicine Technique}
\date{}
% \author{}

\begin{document}
\maketitle
\noindent
\textbf{\large Abstract}:\\ \\ \large Molecular imaging is emerging as an exciting new discipline that deals with imaging of disease on a cellular or genetic level. Nuclear medicine has traditionally focused on non-invasive imaging of in vivo physiology using radiolabeled tracers. As such, molecular imaging has its roots in nuclear medicine and in many ways is a direct extension of this field. The myriad of biological processes that may be targeted for molecular nuclear imaging can be grouped into direct and indirect strategies, depending on the type of imaging probe. The direct strategy uses de novo synthesis of molecular probes targeted to a specific molecular marker such as a receptor, transporter, or enzyme. For each novel target, new radiolabeled compounds are required as well as characterization of their detection sensitivity, interaction specificity, pharmacokinetics of delivery, and signal-to-noise ratio. The indirect strategy entails the use of a pre targeting molecule that is subsequently activated upon occurrence of a specific molecular event, which in turn is targeted by a specific molecular radio probe. Reporter gene imaging falls into this category and provides a rapid and convenient tool to monitor gene expression by yielding a phenotype that is readily imaged upon expression. The remarkable efforts currently focused on the molecular nuclear technology signify its importance and wide range of application. With continued improvements in instrumentation, identification of novel targets, and design of better radio probes, molecular nuclear imaging promises to play an increasingly important role in disease diagnosis and therapy.
\noindent
\\ \\ \textbf{\large Table of content:}
\\ \\Introduction \\ What is nuclear medicine \\ Nuclear medicine scans \\ Nuclear medicine Technique \\Diagnostic medical imaging \\ radioactive tracers \\  Benefits \\ Risk \\ Conclusion \\ Reference \\ \\ \textbf{\large Introduction:} \\ \\ Nuclear medicine is a medical specialty involving the application of radioactive substances in the diagnosis and treatment of disease. Nuclear medicine imaging, in a sense, is "radiology done inside out" or "endoradiology" because it records radiation emitting from within the body rather than radiation that is generated by external sources like X-rays. In addition, nuclear medicine scans differ from radiology, as the emphasis is not on imaging anatomy, but on the function. For such reason, it is called a physiological imaging modality. Single photon emission computed tomography (SPECT) and positron emission tomography (PET) scans are the two most common imaging modalities in nuclear medicine.Nuclear medicine uses small amounts of radioactive material called radiotracers. Doctors use nuclear medicine to diagnose, evaluate, and treat various diseases. These include cancer, heart disease, gastrointestinal, endocrine, or neurological disorders, and other conditions. Nuclear medicine exams pinpoint molecular activity. This gives them the potential to find disease in its earliest stages. They can also show whether you are responding to treatment.
\\ \noindent
\\ \textbf{\large Diagnosis:}

\noindent Nuclear medicine is noninvasive. Except for intravenous injections, it is usually painless. These tests use radioactive materials called radiopharmaceuticals or radiotracers to help diagnose and assess medical conditions.

\noindent Radiotracers are molecules linked to, or "labeled" with, a small amount of radioactive material. They accumulate in tumors or regions of inflammation. They can also bind to specific proteins in the body. The most common radiotracer is F-18 fluorodeoxyglucose (FDG), a molecule similar to glucose. Cancer cells are more metabolically active and may absorb glucose at a higher rate. This higher rate can be seen on PET scans. This allows your doctor to detect disease before it may be seen on other imaging tests. FDG is just one of many radiotracers in use or in development.

\noindent You will usually receive the radiotracer in an injection. Or you may swallow it or inhale it as a gas, depending on the exam. It accumulates in the area under examination. A special camera detects gamma ray emissions from the radiotracer. The camera and a computer produce pictures and supply molecular information.

 
\noindent Many imaging centers combine nuclear medicine images with computed tomography (CT) or magnetic resonance imaging (MRI) to produce special views. Doctors call this image fusion or co-registration. Image fusion allows the doctor to connect and interpret information from two different exams on one image. This leads to more precise information and a more exact diagnosis. Single photon emission CT/CT (SPECT/CT) and positron emission tomography/CT (PET/CT) units can perform both exams at the same time. PET/MRI is an emerging imaging technology. It is not currently available everywhere.
\noindent 
\\ \\ \textbf{\large Therapy :}
\\ Nuclear medicine also offers therapeutic procedures, such as radioactive iodine (I-131) therapy that use small amounts of radioactive material to treat cancer and other medical conditions affecting the thyroid gland, as well as treatments for other cancers and medical conditions.

\noindent Non-Hodgkin's lymphoma patients who do not respond to chemotherapy may undergo radioimmunotherapy (RIT).

\noindent Radioimmunotherapy (RIT) is a personalized cancer treatment that combines radiation therapy with the targeting ability of immunotherapy, a treatment that mimics cellular activity in the body's immune system. See the Radioimmunotherapy (RIT) page for more information.\\


\\ \\ \noindent
\textbf{\large What is nuclear medicine :} \\ \\ Nuclear medicine is a specialized area of radiology that uses very small amounts of radioactive materials, or radiopharmaceuticals, to examine organ function and structure. Nuclear medicine imaging is a combination of many different disciplines. These include chemistry, physics, mathematics, computer technology, and medicine. This branch of radiology is often used to help diagnose and treat abnormalities very early in the progression of a disease, such as thyroid cancer.

\noindent Because X-rays pass through soft tissue, such as intestines, muscles, and blood vessels, these tissues are difficult to visualize on a standard X-ray, unless a contrast agent is used. This allows the tissue to be seen more clearly. Nuclear imaging enables visualization of organ and tissue structure as well as function. The extent to which a radiopharmaceutical is absorbed, or "taken up," by a particular organ or tissue may indicate the level of function of the organ or tissue being studied. Thus, diagnostic X-rays are used primarily to study anatomy. Nuclear imaging is used to study organ and tissue function.

\noindent A tiny amount of a radioactive substance is used during the procedure to assist in the exam. The radioactive substance, called a radionuclide (radiopharmaceutical or radioactive tracer), is absorbed by body tissue. Several different types of radionuclides are available. These include forms of the elements technetium, thallium, gallium, iodine, and xenon. The type of radionuclide used will depend on the type of study and the body part being studied.

\noindent After the radionuclide has been given and has collected in the body tissue under study, radiation will be given off. This radiation is detected by a radiation detector. The most common type of detector is the gamma camera. Digital signals are produced and stored by a computer when the gamma camera detects the radiation.

\noindent By measuring the behavior of the radionuclide in the body during a nuclear scan, the healthcare provider can assess and diagnose various conditions, such as tumors, infections, hematomas, organ enlargement, or cysts. A nuclear scan may also be used to assess organ function and blood circulation.

\noindent The areas where the radionuclide collects in greater amounts are called "hot spots." The areas that do not absorb the radionuclide and appear less bright on the scan image are referred to as "cold spots."

\noindent In planar imaging, the gamma camera remains stationary. The resulting images are two-dimensional (2D). Single photon emission computed tomography, or SPECT, produces axial "slices" of the organ in question because the gamma camera rotates around the patient. These slices are similar to those performed by a CT scan. In certain instances, such as PET scans, three-dimensional (3D) images can be performed using the SPECT data.\\

\\ \\\noindent  \textbf{\large  Nuclear medicine scans :} \\ \\ Scans are used to diagnose many medical conditions and diseases. Some of the more common tests include the following:\\

\noindent \textbf{\large  Renal scans.} These are used to examine the kidneys and to find any abnormalities. These include abnormal function or obstruction of the renal blood flow.

\noindent \textbf{\large Thyroid scans.} These are used to evaluate thyroid function or to better evaluate a thyroid nodule or mass.

\noindent \textbf{\large Bone scans.} These are used to evaluate any degenerative and/or arthritic changes in the joints, to find bone diseases and tumors, and/or to determine the cause of bone pain or inflammation.

\noindent \textbf{\large Gallium scans.} These are used to diagnose active infectious and/or inflammatory diseases, tumors, and abscesses.

\noindent \textbf{\large Heart scans.} These are used to identify abnormal blood flow to the heart, to determine the extent of the damage of the heart muscle after a heart attack, and/or to measure heart function.

\noindent \textbf{\large Brain scans.} These are used to investigate problems within the brain and/or in the blood circulation to the brain.

\noindent \textbf{\large Breast scans.} These are often used in conjunction with mammograms to locate cancerous tissue in the breast.




% To change the title from References to Bibliography:
% \renewcommand\refname{}

% \bibliographystyle{plainnat} % or try abbrvnat or unsrtnat
% \bibliography{} % refers to example.bib

\end{document}