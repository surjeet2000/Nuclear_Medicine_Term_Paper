\documentclass[12pt]{article}
% \usepackage[utf8x]{inputenc}

% \usepackage{natbib}

\title{Nuclear Medicine Technique}
\date{}
% \author{}

\begin{document}
\maketitle
\noindent
\textbf{\large Abstract}:\\ \\ \large Molecular imaging is emerging as an exciting new discipline that deals with imaging of disease on a cellular or genetic level. Nuclear medicine has traditionally focused on non-invasive imaging of in vivo physiology using radiolabeled tracers. As such, molecular imaging has its roots in nuclear medicine and in many ways is a direct extension of this field. The myriad of biological processes that may be targeted for molecular nuclear imaging can be grouped into direct and indirect strategies, depending on the type of imaging probe. The direct strategy uses de novo synthesis of molecular probes targeted to a specific molecular marker such as a receptor, transporter, or enzyme. For each novel target, new radiolabeled compounds are required as well as characterization of their detection sensitivity, interaction specificity, pharmacokinetics of delivery, and signal-to-noise ratio. The indirect strategy entails the use of a pre targeting molecule that is subsequently activated upon occurrence of a specific molecular event, which in turn is targeted by a specific molecular radio probe. Reporter gene imaging falls into this category and provides a rapid and convenient tool to monitor gene expression by yielding a phenotype that is readily imaged upon expression. The remarkable efforts currently focused on the molecular nuclear technology signify its importance and wide range of application. With continued improvements in instrumentation, identification of novel targets, and design of better radio probes, molecular nuclear imaging promises to play an increasingly important role in disease diagnosis and therapy.
\noindent
\\ \\ \textbf{\large Table of content:}
\\ \\Introduction \\ What is nuclear medicine \\ Nuclear medicine Technique \\  Benefits \\ Risk \\ \\ \textbf{\large Introduction:} \\ \\ Nuclear medicine uses small amounts of radioactive material called radiotracers. Doctors use nuclear medicine to diagnose, evaluate, and treat various diseases. These include cancer, heart disease, gastrointestinal, endocrine, or neurological disorders, and other conditions. Nuclear medicine exams pinpoint molecular activity. This gives them the potential to find disease in its earliest stages. They can also show whether you are responding to treatment.
\\ \noindent
\\ \textbf{\large Diagnosis:}

\noindent Nuclear medicine is noninvasive. Except for intravenous injections, it is usually painless. These tests use radioactive materials called radiopharmaceuticals or radiotracers to help diagnose and assess medical conditions.

\noindent Radiotracers are molecules linked to, or "labeled" with, a small amount of radioactive material. They accumulate in tumors or regions of inflammation. They can also bind to specific proteins in the body. The most common radiotracer is F-18 fluorodeoxyglucose (FDG), a molecule similar to glucose. Cancer cells are more metabolically active and may absorb glucose at a higher rate. This higher rate can be seen on PET scans. This allows your doctor to detect disease before it may be seen on other imaging tests. FDG is just one of many radiotracers in use or in development.

\noindent You will usually receive the radiotracer in an injection. Or you may swallow it or inhale it as a gas, depending on the exam. It accumulates in the area under examination. A special camera detects gamma ray emissions from the radiotracer. The camera and a computer produce pictures and supply molecular information.

 
\noindent Many imaging centers combine nuclear medicine images with computed tomography (CT) or magnetic resonance imaging (MRI) to produce special views. Doctors call this image fusion or co-registration. Image fusion allows the doctor to connect and interpret information from two different exams on one image. This leads to more precise information and a more exact diagnosis. Single photon emission CT/CT (SPECT/CT) and positron emission tomography/CT (PET/CT) units can perform both exams at the same time. PET/MRI is an emerging imaging technology. It is not currently available everywhere.
\noindent 
\\ \textbf{\large Therapy :}
\\ Nuclear medicine also offers therapeutic procedures, such as radioactive iodine (I-131) therapy that use small amounts of radioactive material to treat cancer and other medical conditions affecting the thyroid gland, as well as treatments for other cancers and medical conditions.

\noindent Non-Hodgkin's lymphoma patients who do not respond to chemotherapy may undergo radioimmunotherapy (RIT).

\noindent Radioimmunotherapy (RIT) is a personalized cancer treatment that combines radiation therapy with the targeting ability of immunotherapy, a treatment that mimics cellular activity in the body's immune system. See the Radioimmunotherapy (RIT) page for more information.




% To change the title from References to Bibliography:
% \renewcommand\refname{}

% \bibliographystyle{plainnat} % or try abbrvnat or unsrtnat
% \bibliography{} % refers to example.bib

\end{document}