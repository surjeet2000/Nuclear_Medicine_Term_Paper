\documentclass[12pt]{article}
% \usepackage[utf8x]{inputenc}

% \usepackage{natbib}

\title{Nuclear Medicine Technique}
\date{}
% \author{}

\begin{document}
\maketitle
\noindent
\textbf{\large Abstract}:\\ \\ \large Molecular imaging is emerging as an exciting new discipline that deals with imaging of disease on a cellular or genetic level. Nuclear medicine has traditionally focused on non-invasive imaging of in vivo physiology using radiolabeled tracers. As such, molecular imaging has its roots in nuclear medicine and in many ways is a direct extension of this field. The myriad of biological processes that may be targeted for molecular nuclear imaging can be grouped into direct and indirect strategies, depending on the type of imaging probe. The direct strategy uses de novo synthesis of molecular probes targeted to a specific molecular marker such as a receptor, transporter, or enzyme. For each novel target, new radiolabeled compounds are required as well as characterization of their detection sensitivity, interaction specificity, pharmacokinetics of delivery, and signal-to-noise ratio. The indirect strategy entails the use of a pre targeting molecule that is subsequently activated upon occurrence of a specific molecular event, which in turn is targeted by a specific molecular radio probe. Reporter gene imaging falls into this category and provides a rapid and convenient tool to monitor gene expression by yielding a phenotype that is readily imaged upon expression. The remarkable efforts currently focused on the molecular nuclear technology signify its importance and wide range of application. With continued improvements in instrumentation, identification of novel targets, and design of better radio probes, molecular nuclear imaging promises to play an increasingly important role in disease diagnosis and therapy.
\noindent
\\ \\ \large Table of content:
\\ \\Introduction \\ What is nuclear medicine \\ Nuclear medicine Technique




% To change the title from References to Bibliography:
% \renewcommand\refname{}

% \bibliographystyle{plainnat} % or try abbrvnat or unsrtnat
% \bibliography{} % refers to example.bib

\end{document}